\documentclass[a4paper]{article}
\usepackage[USenglish,ngerman]{babel}
\usepackage{lmodern}
\usepackage[utf8x]{inputenc}
\usepackage[T1]{fontenc}
\usepackage{geometry}
\geometry{a4paper, top=25mm, left=25mm, right=25mm, bottom=13mm, headsep=4mm, footskip=4mm}
\usepackage{lastpage}
\usepackage{amsmath}
\usepackage{amssymb}
\usepackage{stmaryrd}
\usepackage{pdflscape}
\usepackage{listings}
\usepackage{courier}
\usepackage{graphicx}
\author{Eric Lanfer}
\usepackage{fancyhdr}
\pagestyle{fancy}
\fancyhf{}
\fancyhead[L]{FizzBuzz}
\fancyhead[C]{}
\fancyhead[R]{Codingchaos}
\renewcommand{\headrulewidth}{0.4pt}
\fancyfoot[C]{Seite \thepage ~von \pageref{LastPage}}
\renewcommand{\footrulewidth}{0.4pt}
\begin{document}
\section*{FizzBuzz}
In der Aufgabe FizzBuzz geht es darum eine Integer Zahl entgegen zu nehmen, euer Programm soll dann bis zu dieser Zahl hinaufzählen. Dabei soll bei jeder Zahl die ganzzahlig durch 3 teilbar ist anstatt der Zahl das Wort "Fizz" ausgegeben werden, bei jeder Zahl die durch 5 teilbar ist das Wort "Buzz" und bei Zahlen die durch 3 und 5 teilbar sind das Wort "FizzBuzz".
\section*{Die Funktion}
Für jede Sprache gilt: Die "gedruckten Zahlen" sollen von der Funktion als String zurück gegeben werden. Als Eingabeparameter soll eine Integer Zahl entgegen genommen werden.
\subsection*{Python}
\begin{lstlisting}[language=Python]
	def fizzbuzz(n):
\end{lstlisting}
\subsection*{Java}
\begin{lstlisting}[language=Java]
	class Fizzbuzz {
		public static String fizzbuzz (int n) {};
	}
\end{lstlisting}
\subsection*{Ruby}
\begin{lstlisting}[language=Ruby]
	def fizzbuzz (n)
\end{lstlisting}
\subsection*{C}
\begin{lstlisting}[language=C]
	int fizzbuzz (int n)
\end{lstlisting}
\newpage
\onecolumn
\section*{Lösungen}
\subsection*{Python}
\begin{lstlisting}[language=Python]
	def fizzbuzz(n):
\end{lstlisting}
\subsection*{Java}
\begin{lstlisting}[language=Java]
	class Fizzbuzz {
		public static String fizzbuzz (int n) {};
	}
\end{lstlisting}
\subsection*{Ruby}
\begin{lstlisting}[language=Ruby]
	def fizzbuzz (n)
\end{lstlisting}
\subsection*{C}
\begin{lstlisting}[language=C]
	int fizzbuzz (int n)
\end{lstlisting}
\end{document}