\documentclass[a4paper]{article}
\usepackage[USenglish,ngerman]{babel}
\usepackage{lmodern}
\usepackage[utf8x]{inputenc}
\usepackage[T1]{fontenc}
\usepackage{geometry}
\geometry{a4paper, top=25mm, left=25mm, right=25mm, bottom=13mm, headsep=4mm, footskip=4mm}
\usepackage{lastpage}
\usepackage{amsmath}
\usepackage{amssymb}
\usepackage{stmaryrd}
\usepackage{pdflscape}
\usepackage{listings}
\usepackage{tikz}
\usepackage{courier}
\usepackage{graphicx}
\author{Jochen Saalfeld}
\usepackage{fancyhdr}
\pagestyle{fancy}
\fancyhf{}
\fancyhead[L]{Vier Gewinnt}
\fancyhead[C]{}
\fancyhead[R]{Codingchaos}
\renewcommand{\headrulewidth}{0.4pt}
\fancyfoot[C]{Seite \thepage ~von \pageref{LastPage}}
\renewcommand{\footrulewidth}{0.4pt}
\begin{document}
\section*{Vier Gewinnt}
Ihr seid es leid händisch zu prüfen, wer bei vier gewinnt gewonnen hat, also habt ihr euch dazu entschieden ein Funktion zu bauen, die das für euch prüft. Dieser Funktion gebt ihr einen String der Form:\\
\texttt{[['-','-','-','-','-','-','-'],}\\
\texttt{['-','-','-','-','-','-','-'],}\\
\texttt{['-','-','-','R','R','R','R'],}\\
\texttt{['-','-','-','Y','Y','R','Y'],}\\
\texttt{['-','-','-','Y','R','Y','Y'],}\\
\texttt{['-','-','Y','Y','R','R','R']];}\\
\\
Das \texttt{-} bedeutet, dass an dieser Stelle kein Stein liegt, das \texttt{R} bedeutet, dass hier ein roter Stein liegt und das \texttt{Y} bedeutet, dass hier ein gelber Stein liegt.\\
In dem oben stehenden Beispiel hätte der Spieler mit den roten Steinen gewonnen, da er eine horizontale Linie geformt hat.\\
Es gelten die allgemeinen Regeln für Vier Gewinnt. Es darf also kein Stein nicht auf einem anderen Liegen oder er muss auf dem Boden liegen. Der Spieler hat gewonnen, der zuerst vier Steine in eine Reihe gebracht hat, dabei ist egal ob Vertikal, horizontal oder diagonal.\\
\subsection*{Zu beachtende Fälle}
\paragraph{Rot Gewinnt:} Falls rot gewinnt, solltet ihr \texttt{R} zurück geben.
\paragraph{Gelb Gewinnt:} Falls gelb gewinnt, solltet ihr \texttt{Y} zurück geben.
\paragraph{Niemand Gewinnt:} Falls niemand gewonnen hat, solltet ihr \texttt{NULL} zurück geben.
\paragraph{Zwei \glqq gewinnen\grqq:} Falls \texttt{R} und \texttt{Y} vier Steine korrekt angeordnet haben, solltet ihr \texttt{ERROR} zurückgeben.
\subsection*{Python}
\begin{lstlisting}[language=Python]
	def fourinarow(game):
\end{lstlisting}
\subsection*{Java}
\begin{lstlisting}[language=Java]
	class FourInARow {
		public static char forinarow (String game) {};
	}
\end{lstlisting}
\subsection*{Ruby}
\begin{lstlisting}[language=Ruby]
	def fourinarow (game)
\end{lstlisting}
\subsection*{C}
\begin{lstlisting}[language=C]
	char fourinarow (String game)
\end{lstlisting}
\newpage
\onecolumn
\section*{Lösungen}
\begin{lstlisting}[language=Python]
	def fourinarow(game):
\end{lstlisting}
\subsection*{Java}
\begin{lstlisting}[language=Java]
	class FourInARow {
		public static char forinarow (String game) {};
	}
\end{lstlisting}
\subsection*{Ruby}
\begin{lstlisting}[language=Ruby]
	def fourinarow (game)
\end{lstlisting}
\subsection*{C}
\begin{lstlisting}[language=C]
	char fourinarow (String game)
\end{lstlisting}
\end{document}