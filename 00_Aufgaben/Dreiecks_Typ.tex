\documentclass[a4paper]{article}
\usepackage[USenglish,ngerman]{babel}
\usepackage{lmodern}
\usepackage[utf8x]{inputenc}
\usepackage[T1]{fontenc}
\usepackage{geometry}
\geometry{a4paper, top=25mm, left=25mm, right=25mm, bottom=13mm, headsep=4mm, footskip=4mm}
\usepackage{lastpage}
\usepackage{amsmath}
\usepackage{amssymb}
\usepackage{stmaryrd}
\usepackage{pdflscape}
\usepackage{listings}
\usepackage{courier}
\usepackage{graphicx}
\author{Jochen Saalfeld}
\usepackage{fancyhdr}
\pagestyle{fancy}
\fancyhf{}
\fancyhead[L]{Dreieckstyp}
\fancyhead[C]{}
\fancyhead[R]{Codingchaos}
\renewcommand{\headrulewidth}{0.4pt}
\fancyfoot[C]{Seite \thepage ~von \pageref{LastPage}}
\renewcommand{\footrulewidth}{0.4pt}
\begin{document}
\section*{Dreieckstypen}
Es gibt 3 verschieden Formen von Dreiecken. In dieser Aufgabe sollt ihr die drei Seitenlängen eines Dreiecks $(a,b,c)$ in irgendeiner Reihenfolge aufnehmen und den Typen des Dreiecks bestimmen.
\paragraph{Spitzes Dreieck:} Wenn alle Winkel im Dreieck unter 90° groß sind, ist es ein Spitzwinkliges Dreieck und die Funktion sollte \texttt{1} zurück geben.
\paragraph{Rechtwinkliges Dreieck:} Sobald ein Winkel im Dreieck 90° groß ist, ist es ein rechtwinkles Dreieck und die Funktion sollte \texttt{2} zurück geben.
\paragraph{Stumpfwinkliges Dreieck:} Wenn ein Winkel über 90° groß ist, ist es ein stumpfes Dreieck und die Funktion sollte \texttt{3} zurück geben.
\paragraph{Übrige Fälle:} Wenn ein Winkel des Dreiecks 180° groß ist, wird das Dreieck zur Linie und es soll \texttt{0} zurück gegeben werden.
\section*{Die Funktion}
Für jede Sprache gilt: Es gibt einen Rückgabewert der Funktion vom Typ Integer und es gibt 3 Eingabewerte vom Typ Double.
\subsection*{Python}
\begin{lstlisting}[language=Python]
	def triangle_type(a, b, c):
\end{lstlisting}
\subsection*{Java}
\begin{lstlisting}[language=Java]
	class Traingle {
		public static int triangle_type (int a, int b, int c) {};
	}
\end{lstlisting}
\subsection*{Ruby}
\begin{lstlisting}[language=Ruby]
	def triangle_type (a, b, c)
\end{lstlisting}
\subsection*{C}
\begin{lstlisting}[language=C]
	int triangle_type (int a, int b, int c)
\end{lstlisting}
\newpage
\onecolumn
\section*{Lösungen}
\subsection*{Python}
\begin{lstlisting}[language=Python]
	def triangle_type(a, b, c):
\end{lstlisting}
\subsection*{Java}
\begin{lstlisting}[language=Java]
	class Traingle {
		public static int triangle_type (int a, int b, int c) {};
	}
\end{lstlisting}
\subsection*{Ruby}
\begin{lstlisting}[language=Ruby]
	def triangle_type (a, b, c)
\end{lstlisting}
\subsection*{C}
\begin{lstlisting}[language=C]
	int triangle_type (int a, int b, int c)
\end{lstlisting}
\end{document}