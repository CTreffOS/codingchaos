\documentclass[a4paper]{article}
\usepackage[USenglish,ngerman]{babel}
\usepackage{lmodern}
\usepackage[utf8x]{inputenc}
\usepackage[T1]{fontenc}
\usepackage{geometry}
\geometry{a4paper, top=25mm, left=25mm, right=25mm, bottom=13mm, headsep=4mm, footskip=4mm}
\usepackage{lastpage}
\usepackage{amsmath}
\usepackage{amssymb}
\usepackage{stmaryrd}
\usepackage{pdflscape}
\usepackage{listings}
\usepackage{tikz}
\usepackage{courier}
\usepackage{graphicx}
\author{Jochen Saalfeld}
\usepackage{fancyhdr}
\pagestyle{fancy}
\fancyhf{}
\fancyhead[L]{Rechteck in Quadrate}
\fancyhead[C]{}
\fancyhead[R]{Codingchaos}
\renewcommand{\headrulewidth}{0.4pt}
\fancyfoot[C]{Seite \thepage ~von \pageref{LastPage}}
\renewcommand{\footrulewidth}{0.4pt}
\begin{document}
\section*{Rechteck in Quadrate}
Die Folgende Zeichnung gibt euch ein Idee, was passieren soll:\\
\begin{tikzpicture}[scale=2.0]
	\coordinate (0_0) at (0,0);
	\coordinate (1_0) at (1,0);
	\coordinate (2_0) at (2,0);
	\coordinate (3_0) at (3,0);
	\coordinate (4_0) at (4,0);
	\coordinate (5_0) at (5,0);
	
	\coordinate (0_1) at (0,1);
	\coordinate (1_1) at (1,1);
	\coordinate (2_1) at (2,1);
	\coordinate (3_1) at (3,1);
	\coordinate (4_1) at (4,1);
	\coordinate (5_1) at (5,1);
	
	\coordinate (0_2) at (0,2);
	\coordinate (1_2) at (1,2);
	\coordinate (2_2) at (2,2);
	\coordinate (3_2) at (3,2);
	\coordinate (4_2) at (4,2);
	\coordinate (5_2) at (5,2);
	
	\coordinate (0_3) at (0,3);
	\coordinate (1_3) at (1,3);
	\coordinate (2_3) at (2,3);
	\coordinate (3_3) at (3,3);
	\coordinate (4_3) at (4,3);
	\coordinate (5_3) at (5,3);
	
	\fill [orange] (4_0) rectangle (5_1);
	\fill [yellow] (3_0) rectangle (4_1);
	\fill [pink] (3_1) rectangle (5_3);
	\fill [blue] (0_0) rectangle (3_3);
	
	\draw[-] (0_0) -- (5_0);
	\draw[-] (0_1) -- (5_1);	 
	\draw[-] (0_2) -- (5_2);
	\draw[-] (0_3) -- (5_3);
	
	\draw[-] (0_0) -- (0_3);
	\draw[-] (1_0) -- (1_3);
	\draw[-] (2_0) -- (2_3);	
	\draw[-] (3_0) -- (3_3);
	\draw[-] (4_0) -- (4_3);
	\draw[-] (5_0) -- (5_3);	
	
	\node (first) at (1.5,1.5) {3\^{}3=9};
	\node (second) at (4,2) {2\^{}2=4};
	\node (third) at (3.5,0.5) {1\^{}1 = 1};
	\node (third) at (4.5,0.5) {1\^{}1 = 1};
\end{tikzpicture}\\\\
Diese Zeichnung soll in einen Algorithmus Übersetzt werden. Es gibt dabei zwei Eingabeparameter \texttt{length} und \texttt{height}. Zurückgegeben werden sollen, in absteigender Reihenfolge, die Größen der darin enthaltenden Quadrate.
\paragraph{Beispiel}~\\
\texttt{SquareInRectangle(5, 3) returns [3, 2, 1, 1]}\\
\texttt{SquareInRectangle(3, 5) returns [3, 2, 1, 1]}
\subsection*{Python}
\begin{lstlisting}[language=Python]
	def SquareInRectangle(length, height):
\end{lstlisting}
\subsection*{Java}
\begin{lstlisting}[language=Java]
	class SquareInRectangle {
		public static int [] SquareInRectangle (int length, int height) {};
	}
\end{lstlisting}
\subsection*{Ruby}
\begin{lstlisting}[language=Ruby]
	def triangle_type (length, height)
\end{lstlisting}
\subsection*{C}
\begin{lstlisting}[language=C]
	int [] triangle_type (int length, int height)
\end{lstlisting}
\newpage
\onecolumn
\section*{Lösungen}
\subsection*{Python}
\begin{lstlisting}[language=Python]
	def triangle_type(a, b, c):
\end{lstlisting}
\subsection*{Java}
\begin{lstlisting}[language=Java]
	class Traingle {
		public static int triangle_type (int a, int b, int c) {};
	}
\end{lstlisting}
\subsection*{Ruby}
\begin{lstlisting}[language=Ruby]
	def triangle_type (a, b, c)
\end{lstlisting}
\subsection*{C}
\begin{lstlisting}[language=C]
	int triangle_type (int a, int b, int c)
\end{lstlisting}
\end{document}